\documentclass[12pt,a4paper]{article}
\usepackage[utf8]{inputenc}
\DeclareUnicodeCharacter{2265}{\ensuremath{\geq}}
\usepackage[T1]{fontenc}
\usepackage{lmodern}
\usepackage{geometry}
\usepackage{enumitem}
\usepackage{hyperref}
\usepackage{titlesec}
\usepackage{parskip}
\usepackage{longtable}
\usepackage{graphicx}

\geometry{margin=2.5cm}
\titleformat{\section}{\normalfont\Large\bfseries}{\thesection}{1em}{}
\titleformat{\subsection}{\normalfont\large\bfseries}{\thesubsection}{1em}{}

\title{Requirements Document\\\large WeatherNow Calendar Integration}
\author{
  Pascal Putz \\ \texttt{pascal.putz@study.thws.de} \\ 5123135
  \and
  [Team Member 2] \\ \texttt{email@domain} \\ 7654321
  % bis zu vier Teammitglieder
}
\date{April 2025}

\begin{document}
\maketitle

\section{Author Information}
\begin{longtable}{|p{5cm}|p{6cm}|p{3cm}|}
\hline
\textbf{Name} & \textbf{Email} & \textbf{Matrikelnummer} \\
\hline
Pascal Putz & pascal.putz@study.thws.de & 5123135 \\
\hline
[Team Member 2] & email@domain & 7654321 \\
\hline
% Weitere Teammitglieder bis max. 4
\end{longtable}



\section{Project Overview}
\textbf{Background.} In modern daily life, accurate weather information is essential for planning activities, travel, and commuting. However, most weather apps do not consider a user’s personal schedule, leading to suboptimal planning.

\textbf{Problem Statement.} Users must manually cross‐check weather forecasts with their personal calendars to decide what to do on free days. This manual process is time-consuming and error-prone.

\textbf{Solution.} WeatherNow Calendar Integration is a React-based frontend application that merges real-time and forecasted weather data from the OpenWeatherMap API with free/busy information from the Google Calendar API. By selecting a date in an interactive calendar, users receive tailored activity suggestions based on expected weather conditions and their actual availability.

\textbf{Objectives.}
\begin{itemize}[nosep]
  \item Automate the correlation of weather forecasts with user availability.
  \item Provide clear, actionable activity recommendations.
  \item Offer an intuitive, responsive interface for mobile and desktop.
  \item Demonstrate proficiency in React, RESTful API integration, and containerized deployment.
\end{itemize}

\textbf{Target Audience.}  
Travelers, commuters, students, and outdoor enthusiasts who wish to optimize their free time around weather conditions without manual cross-referencing.


\section{Key Features}
\subsection*{Must‐Have Requirements}
\begin{itemize}[nosep]
  \item \textbf{City Search \& Current Weather.}  
    Users enter a city name; the app validates input, calls 
    \texttt{/data/2.5/weather?q=City\&units=metric}, and displays: temperature, humidity, wind speed/direction, pressure, description, and icon.

  \item \textbf{Five‐Day Forecast.}  
    Retrieves 3-hour‐interval forecasts from 
    \texttt{/data/2.5/forecast?q=City\&units=metric}, computes daily high/low,
    and presents both detailed and summary views.

  \item \textbf{Geolocation Lookup.}  
    Upon permission, the browser’s Geolocation API provides coords; weather
    is fetched via 
    \texttt{/data/2.5/weather?lat=\dots\&lon=\dots} without manual entry.

  \item \textbf{Unit Toggle.}  
    Allows switching between Celsius and Fahrenheit. Temperatures update
    dynamically by re-fetching with \texttt{\&units=metric} or 
    \texttt{\&units=imperial}.

  \item \textbf{Theme Toggle.}  
    Dark and light modes enhance usability in different lighting; preference
    persists via \texttt{localStorage}.

  \item \textbf{Interactive Calendar.}  
    Embeds a calendar widget (e.g., \texttt{react-calendar}). Users select any
    date to trigger weather analysis.

  \item \textbf{Activity Suggestions.}  
    For the chosen date, forecast (or historical) data is analyzed against
    predefined rules (e.g., sunny $\rightarrow$ hiking; rainy $\rightarrow$
    museum) and recommendations are displayed.

  \item \textbf{Error Handling.}  
    All API calls use try/catch. The UI shows toast notifications for errors
    (“City not found,” “Permission denied,” “Network error”).

  \item \textbf{Responsive UI.}  
    CSS Grid/Flexbox ensure layouts adapt for mobile (<600px), tablet (600–1024px),
    and desktop (>1024px).

  \item \textbf{Client‐Side Navigation.}  
    React Router manages routes for Home, Forecast, and Calendar views,
    updating the URL accordingly (e.g., \texttt{/forecast}).
\end{itemize}

\subsection*{Nice‐to‐Have Requirements}
\begin{itemize}[nosep]
  \item Store and display recently searched cities.
  \item Synchronize multiple Google calendars (work, personal).
  \item Localize UI and weather descriptions (EN/DE).
  \item Lookup historical weather for past dates.
  \item Push notifications for severe weather or upcoming free days.
\end{itemize}



\section{User Roles and Interactions}
\textbf{Primary Role:} End User.

\subsection*{Personas}
\begin{itemize}[nosep]
  \item \textbf{Anna (28, Hiker).} Uses app on weekends to plan outdoor trips.
  \item \textbf{Max (35, Commuter).} Checks rain forecasts before biking.
  \item \textbf{Lisa (22, Student).} Schedules museum visits on free days.
\end{itemize}

\subsection*{Interaction Flow}
\begin{enumerate}[nosep]
  \item User lands on Home; may grant geolocation.
  \item Home displays current weather for location or default city.
  \item User navigates to Forecast via menu to see five‐day outlook.
  \item User opens Calendar, clicks a date; weather data and suggestions load.
  \item User toggles units/theme; UI re-renders accordingly.
  \item Errors are shown inline or via toast if operations fail.
\end{enumerate}




\section{User Stories / Use Cases}
\begin{longtable}{|p{1cm}|p{3cm}|p{10cm}|}
\hline
\textbf{ID} & \textbf{Title} & \textbf{Description} \\
\hline
US1 & Search City & As a user, I enter a city name to retrieve and view current weather conditions. \\
\hline
US2 & View Forecast & As a user, I request a five‐day forecast to plan my activities. \\
\hline
US3 & Use My Location & As a user, I allow geolocation so I can see weather for my current position. \\
\hline
US4 & Select Date & As a user, I click a date on the calendar to view weather and activity suggestions. \\
\hline
US5 & Toggle Units & As a user, I switch between Celsius and Fahrenheit for my preference. \\
\hline
US6 & Toggle Theme & As a user, I enable dark mode in low‐light conditions. \\
\hline
US7 & Handle Error & As a user, I receive informative feedback when an API call or permission fails. \\
\hline
\end{longtable}



\section{Non‐Functional Requirements}
\subsection*{Usability}
The UI shall use consistent typography, spacing, and icon sets. Tooltips and labels
ensure discoverability without documentation.

\subsection*{Accessibility}
The application shall conform to WCAG 2.1 AA:
\begin{itemize}[nosep]
  \item Color contrast $\ge 4.5:1$.
  \item Keyboard navigation across all interactive elements.
  \item ARIA roles and labels on custom components.
\end{itemize}

\subsection*{Performance}
\begin{itemize}[nosep]
  \item Initial bundle size $\le$ 200 KB (gzipped).
  \item Time to Interactive (TTI) $\le$ 1.5 s on 3G.
  \item Lazy load calendar and forecast modules.
\end{itemize}

\subsection*{Reliability}
Implement automatic retry (3 attempts) with exponential backoff for transient API errors.
Cache last‐fetched data in \texttt{sessionStorage}.

\subsection*{Security}
\begin{itemize}[nosep]
  \item Enforce HTTPS and HSTS.
  \item Store API keys in environment variables; do not expose in client bundle.
  \item Use OAuth 2.0 for Google Calendar access.
\end{itemize}

\subsection*{Maintainability}
\begin{itemize}[nosep]
  \item Modular component structure: \texttt{components/}, \texttt{services/}, \texttt{styles/}.
  \item Code style enforced via ESLint and Prettier.
  \item Inline JSDoc comments for key functions.
\end{itemize}

\subsection*{Scalability \& Extensibility}
Architecture shall support:
\begin{itemize}[nosep]
  \item Addition of new APIs (e.g., air quality, UV index) via service layer abstraction.
  \item New UI themes or languages with minimal effort.
\end{itemize}



\section{Technology Assumptions}
\begin{itemize}[nosep]
  \item \textbf{Frontend Framework:} React 18 with Hooks and Context API.
  \item \textbf{Styling:} CSS Modules or Tailwind CSS (TBD by team).
  \item \textbf{Routing:} React Router DOM v6 for nested and parameterized routes.
  \item \textbf{Calendar Component:} \texttt{react-calendar} for date selection.
  \item \textbf{Notifications:} \texttt{react-toastify} for user alerts.
  \item \textbf{Build Tools:} Vite for fast development and optimized production builds.
  \item \textbf{Containerization:} Docker multi-stage build; Nginx serves static files.
  \item \textbf{Weather API:} OpenWeatherMap REST endpoints (current, forecast, geocoding).
  \item \textbf{Calendar API:} Google Calendar via OAuth 2.0 (free/busy, events).
\end{itemize}

\section{Project Constraints}
\begin{itemize}[nosep]
  \item OpenWeatherMap free tier: max 60 calls/minute.
  \item Google Calendar API quotas and OAuth consent verification timeline.
  \item Requires explicit user permission for geolocation and calendar access.
  \item No custom backend development; all logic in frontend.
  \item Requirements Document due: April 30, 2025.
  \item \textbf{Project Timeline:}
  \begin{itemize}[nosep]
    \item Requirements: by April 30, 2025
    \item Design Mockups: by May 7, 2025
    \item Implementation: May 8 – July 10, 2025
    \item Testing \& Refinement: July 11 – July 20, 2025
    \item Final Submission: July 30, 2025
  \end{itemize}
\end{itemize}


\section{Acknowledgement of AI Tools}
This document was drafted and refined using GPT-4o based on an outline provided by the authors. The authors critically reviewed and enhanced the output to ensure coherence, completeness, and alignment with course requirements.

\end{document}
