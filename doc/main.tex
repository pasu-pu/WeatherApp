\documentclass[11pt,a4paper]{article}
\usepackage[utf8]{inputenc}
\usepackage[T1]{fontenc}
\usepackage{lmodern}
\usepackage{geometry}
\usepackage{enumitem}
\usepackage{hyperref}
\usepackage{titlesec}
\usepackage{parskip}
\usepackage{longtable}

% Unicode-Zeichen 202F (narrow no-break space) als dünnen Abstand definieren
\DeclareUnicodeCharacter{202F}{\,}

\geometry{margin=2.5cm}
\titleformat{\section}{\normalfont\Large\bfseries}{\thesection}{1em}{}
\titleformat{\subsection}{\normalfont\large\bfseries}{\thesubsection}{1em}{}

\title{Requirements Document xD\\\large WeatherNow Calendar Integration}
\author{
  Pascal Putz \\ \texttt{pascal.putz@study.thws.de} \\ 5123135
  \and
  [Weitere Teammitglieder hier]
}
\date{April 2025}

\begin{document}
\maketitle

\section{Author Information}
\begin{longtable}{|p{5cm}|p{6cm}|p{3cm}|}
\hline
\textbf{Name} & \textbf{Email} & \textbf{Matrikelnummer} \\
\hline
Joel Putz & joel.putz@example.com & 1234567 \\
\hline
% Weitere Teammitglieder hier einfügen
\end{longtable}

\section{Project Overview}
We will build \textbf{WeatherNow Calendar}, a React-based frontend application combining:
\begin{itemize}[nosep]
  \item Real-time and forecasted weather data (OpenWeatherMap API)
  \item User’s schedule and free days (Google Calendar API)
  \item Activity recommendations based on weather and availability
\end{itemize}
Users—such as travelers, commuters and outdoor enthusiasts—can plan free-time activities by selecting a date in their calendar; the app then analyzes weather statistics for that day and suggests suitable indoor or outdoor options.

\section{Key Features}
\subsection*{Must-have}
\begin{itemize}[nosep]
  \item Search by city name: display current weather (temp, wind, humidity, conditions, icon)
  \item 5-day forecast in 3-hour intervals
  \item Geolocation lookup: fetch weather for user’s current position
  \item Toggle units: Celsius / Fahrenheit
  \item Toggle theme: light / dark mode
  \item Interactive calendar: select any date
  \item On date selection: show weather data + activity suggestions
  \item Error handling: invalid city, permission denied, API failures
  \item Responsive UI: mobile and desktop
  \item Navigation: Home, Forecast, Calendar views (React Router)
\end{itemize}

\subsection*{Nice-to-have}
\begin{itemize}[nosep]
  \item Store and recall recent city searches
  \item Sync and overlay multiple Google calendars
  \item Localization of weather descriptions (multi-language)
  \item Historical weather lookup (past dates)
  \item Push notifications for severe weather or upcoming free days
\end{itemize}

\section{User Roles and Interactions}
\textbf{Primary role:} General user  
\textbf{Interactions:}
\begin{itemize}[nosep]
  \item Enter or select a city → view current weather and forecast
  \item Grant calendar access → see free/busy days
  \item Click on a calendar date → fetch and display weather + suggestions
  \item Toggle units and theme via buttons
  \item Navigate via header or menu to Home, Forecast, Calendar
  \item Receive toast/message on errors (e.g., “City not found”)
\end{itemize}

\section{User Stories / Use Cases}
\begin{enumerate}[nosep]
  \item As a user, I want to search for a city name so I can view its current weather.
  \item As a user, I want to see a five-day forecast so I can plan upcoming days.
  \item As a user, I want to grant calendar access so the app knows my free days.
  \item As a user, I want to click a date in my calendar to get weather stats and activity suggestions for that day.
  \item As a user, I want to toggle between °C and °F based on my preference.
  \item As a user, I want dark mode at night to reduce eye strain.
  \item As a user, I want to be notified if an API call fails or if I enter an invalid city.
\end{enumerate}

\section{Non-Functional Requirements}
\begin{itemize}[nosep]
  \item \textbf{Usability:} Clear labels, consistent icons, minimal learning curve
  \item \textbf{Responsiveness:} Fluid layout with CSS Flexbox/Grid for all viewport sizes
  \item \textbf{Accessibility:} ARIA labels, sufficient color contrast, keyboard navigation
  \item \textbf{Performance:} Data fetch + render complete under 1.5\,s on average networks
  \item \textbf{Security:} API keys stored in environment variables, HTTPS only
  \item \textbf{Maintainability:} Modular React components, clear folder structure, documented code
  \item \textbf{Scalability:} Design to add new APIs or features without major refactoring
\end{itemize}

\section{Technology Assumptions}
\begin{itemize}[nosep]
  \item \textbf{Frontend:} React.js, JavaScript (ES6+)
  \item \textbf{Styling:} CSS Modules or Tailwind CSS
  \item \textbf{Routing:} React Router DOM
  \item \textbf{APIs:} 
    \begin{itemize}[nosep]
      \item OpenWeatherMap REST API for current weather, forecast, geocoding 
      \item Google Calendar API for user’s events and free/busy times
    \end{itemize}
  \item \textbf{State Management:} React Context or Redux (optional)
  \item \textbf{Build / Deploy:} Docker container with Nginx serving static build
\end{itemize}

\section{Project Constraints}
\begin{itemize}[nosep]
  \item OpenWeatherMap free tier: max 60 calls/minute
  \item Google Calendar API quotas and OAuth consent screen requirements
  \item Requires explicit user consent for calendar and location access
  \item No user authentication beyond Google OAuth for calendar access
  \item Deadline for Requirements Document: April 30, 2025
\end{itemize}

\section{Acknowledgement of AI Tools}
This document was drafted and refined using GPT-4o based on an outline containing project information. The authors reviewed, revised, and enhanced the GPT-4o output with additional content and edited for clarity and style.

\end{document}
