\documentclass[12pt,a4paper]{article}
\usepackage[utf8]{inputenc}
\DeclareUnicodeCharacter{202F}{\,}  % narrow no-break space
\usepackage[T1]{fontenc}
\usepackage{lmodern}
\usepackage{geometry}
\usepackage{enumitem}
\usepackage{hyperref}
\usepackage{titlesec}
\usepackage{parskip}
\usepackage{longtable}
\usepackage{graphicx}

\geometry{margin=2.5cm}
\titleformat{\section}{\normalfont\Large\bfseries}{\thesection}{1em}{}
\titleformat{\subsection}{\normalfont\large\bfseries}{\thesubsection}{1em}{}

\title{Requirements Document\\\large WeatherNow Calendar Integration}
\author{
  Pascal Putz \\ \texttt{pascal.putz@study.thws.de} \\ 5123135
  \and
  Gunn Kataria \\ \texttt{gunn.kataria@study.thws.de} \\ 9125072
  \and
  Katrina Alex \\ \texttt{katrina.alex@study.thws.de} \\ 9125071
  \and
  Manuel Stöth \\ \texttt{manuel.stoeth@study.thws.de} \\ 5123045
  \and
  Marvin Kraus \\ \texttt{marvin.kraus@study.thws.de} \\ 5123143
  \and
}
\date{April 2025}

\begin{document}
\maketitle

\section{Author Information}
\begin{longtable}{|p{5cm}|p{6cm}|p{3cm}|}
\hline
\textbf{Name} & \textbf{Email} & \textbf{StudyID} \\
\hline
Pascal Putz & pascal.putz@study.thws.de & 5123135 \\
\hline
Gunn Kataria & gunn.kataria@study.thws.de & 9125072 \\
\hline
Katrina Alex & katrina.alex@study.thws.de & 9125071 \\
\hline
Manuel Stöth & manuel.stoeth@study.thws.de & 5123045 \\
\hline
Marvin Kraus & marvin.kraus@study.thws.de & 5123143 \\
\hline
\end{longtable}

%\newpage

\section{Project Overview}

\subsection{Project Description}
WeatherNow Calendar Integration is a web-based application designed to simplify free-time planning by combining real-time weather data with the user’s personal calendar schedule. By leveraging the OpenWeatherMap API and the Google Calendar API, the app allows users to check current and forecasted weather conditions alongside their availability. Based on weather statistics and free days, WeatherNow suggests personalized activities, helping users make the most of their time—whether it’s a sunny outdoor hike or a cozy indoor museum visit. The frontend is built using React.js to ensure a modern, responsive, and smooth user experience across both mobile and desktop devices.

\subsection{Application Goals}
\begin{itemize}[nosep]
  \item Enable users to view accurate current weather and detailed forecasts for any location worldwide.
  \item Integrate the user’s Google Calendar to identify free and busy periods automatically.
  \item Suggest context-aware activities based on weather conditions and user availability.
  \item Provide a user-friendly, accessible, and visually appealing interface with light/dark mode options.
  \item Ensure fast, reliable, and error-tolerant performance for all weather and calendar interactions.
  \item Build a modular and scalable frontend that can be extended with additional APIs or features in the future.
\end{itemize}

\subsection{Target Users}
\begin{itemize}[nosep]
  \item \textbf{Travelers and Tourists} planning outdoor or sightseeing activities.
  \item \textbf{Daily Commuters} preparing for weather disruptions during workdays.
  \item \textbf{Outdoor Enthusiasts} (hikers, cyclists, sports players) relying on forecasts.
  \item \textbf{Busy Professionals and Students} maximizing free time with suitable recommendations.
  \item \textbf{Families and Groups} organizing weekend outings or vacations dependent on weather.
\end{itemize}

\section{Project Complexity}
This project exhibits moderate complexity:
\begin{itemize}[nosep]
  \item \textbf{Multiple Views:} Home, Forecast, Calendar, Settings.
  \item \textbf{API Integration:} Asynchronous calls to two external services.
  \item \textbf{Interactive Widgets:} Calendar component, geolocation lookup.
  \item \textbf{State Management:} React Context (or Redux) for global state.
  \item \textbf{Responsive Design:} Mobile-first layout with CSS Grid/Flexbox.
\end{itemize}

%\newpage

\section{Key Features}

\subsection*{Must-Have}
\begin{itemize}[nosep]
  \item \textbf{City Search \& Current Weather:} Users search a city, see temperature, wind, humidity, conditions and icon.
  \item \textbf{5-Day Forecast:} Forecast in 3-hour intervals with daily high/low summaries.
  \item \textbf{Geolocation Lookup:} Automatically fetch weather for user’s current coordinates.
  \item \textbf{Unit Toggle:} Switch between °C and °F.
  \item \textbf{Theme Toggle:} Light and dark modes, persisted across sessions.
  \item \textbf{Interactive Calendar:} Select any date to trigger analysis.
  \item \textbf{Activity Suggestions:} Recommend indoor/outdoor activities based on weather rules.
  \item \textbf{Error Handling:} Friendly messages for invalid input, denied permissions, or API failures.
  \item \textbf{Responsive UI:} Adapts to mobile, tablet, and desktop breakpoints.
  \item \textbf{Client-Side Navigation:} React Router routes for Home, Forecast, and Calendar.
\end{itemize}

\subsection*{Nice-to-Have}
\begin{itemize}[nosep]
  \item Store and recall recently searched cities.
  \item Sync and overlay multiple Google calendars.
  \item Localize UI text and weather descriptions (e.g., EN/DE).
  \item Lookup historical weather for past dates.
  \item Push notifications for severe weather or upcoming free days.
\end{itemize}

%\newpage

\section{User Roles and Interactions}
\textbf{Primary role:} General user  
The system shall enable users to:
\begin{itemize}[nosep]
  \item Enter or select a city to view weather and forecast.
  \item Grant calendar access to retrieve busy/free days.
  \item Select a date on the calendar to fetch weather and activity suggestions.
  \item Toggle units (°C/°F) and theme (light/dark).
  \item Navigate via header/menu to Home, Forecast, and Calendar.
  \item Receive inline or toast notifications on errors.
\end{itemize}

\section{User Stories / Use Cases}
\begin{longtable}{|p{1cm}|p{3cm}|p{10cm}|}
\hline
\textbf{ID} & \textbf{Title} & \textbf{Description} \\
\hline
US1 & Search City & As a user, I enter a city name to view current weather. \\
\hline
US2 & View Forecast & As a user, I request a five-day forecast to plan ahead. \\
\hline
US3 & Use My Location & As a user, I grant location access to see local weather. \\
\hline
US4 & Select Date & As a user, I click a calendar date to get weather stats and activity suggestions. \\
\hline
US5 & Toggle Units & As a user, I switch between Celsius and Fahrenheit. \\
\hline
US6 & Toggle Theme & As a user, I enable dark mode at night. \\
\hline
US7 & Handle Error & As a user, I get notified on API failures or invalid input. \\
\hline
\end{longtable}
\end{enumerate}

%\newpage

\section{Non-Functional Requirements}
\begin{itemize}[nosep]
  \item \textbf{Usability:} Clear labels, consistent icons, minimal learning curve.
  \item \textbf{Responsiveness:} Fluid layout using CSS Flexbox/Grid for all viewports.
  \item \textbf{Accessibility:} ARIA labels, sufficient contrast, keyboard navigation.
  \item \textbf{Performance:} Data fetch and render complete under 1.5\,s on average networks.
  \item \textbf{Security:} HTTPS only; API keys in environment variables.
  \item \textbf{Maintainability:} Modular React components; documented code.
  \item \textbf{Scalability:} Architecture supports adding new APIs/features with minimal refactoring.
\end{itemize}

\section{Technology Assumptions}
\begin{itemize}[nosep]
  \item \textbf{Frontend:} React.js (ES6+).
  \item \textbf{Styling:} CSS Modules or Tailwind CSS.
  \item \textbf{Routing:} React Router DOM.
  \item \textbf{APIs:} OpenWeatherMap REST, Google Calendar OAuth2.
  \item \textbf{State Management:} React Context or Redux.
  \item \textbf{Build/Deploy:} Docker multi-stage build; Nginx serves static files.
\end{itemize}

\section{Project Constraints}
\begin{itemize}[nosep]
  \item OpenWeatherMap student tier: max 3000 calls/min.
  \item Google Calendar API quotas and OAuth consent requirements.
  \item Requires explicit user permission for location/calendar access.
  \item No custom backend development; all logic in frontend.
  \item Requirements Document due: April 30, 2025.
\end{itemize}

\section{Acknowledgement of AI Tools}
This document was drafted and refined using GPT-4o based on an outline provided by the authors. The authors reviewed and enhanced the output for coherence and alignment with course requirements.

\end{document}
