\documentclass[fontsize=13pt,a4paper]{scrartcl}
\usepackage[utf8]{inputenc}
\DeclareUnicodeCharacter{202F}{\,}  % narrow no-break space
\DeclareUnicodeCharacter{2265}{\ensuremath{\geq}}
\usepackage[T1]{fontenc}
\usepackage{lmodern}
\usepackage{geometry}
\usepackage{enumitem}
\usepackage{hyperref}
\usepackage{titlesec}
\usepackage{parskip}
\usepackage{longtable}
\usepackage{graphicx}

\geometry{margin=2.5cm}
\titleformat{\section}{\normalfont\Large\bfseries}{\thesection}{1em}{}
\titleformat{\subsection}{\normalfont\large\bfseries}{\thesubsection}{1em}{}

\title{Requirements Document\\\large WeatherNow Calendar Integration}
\author{
  Pascal Putz \\ \texttt{pascal.putz@study.thws.de} \\ 5123135
  \and
  Gunn Kataria \\ \texttt{gunn.kataria@study.thws.de} \\ 9125072
  \and
  Katrina Alex \\ \texttt{katrina.alex@study.thws.de} \\ 9125071
  \and
  Manuel Stöth \\ \texttt{manuel.stoeth@study.thws.de} \\ 5123045
  \and
  Marvin Kraus \\ \texttt{marvin.kraus@study.thws.de} \\ 5123143
}
\date{April 2025}

\begin{document}
\maketitle

\section{Author Information}
\begin{longtable}{|p{5cm}|p{6cm}|p{3cm}|}
\hline
\textbf{Name} & \textbf{Email} & \textbf{StudyID} \\
\hline
Pascal Putz & pascal.putz@study.thws.de & 5123135 \\
\hline
Gunn Kataria & gunn.kataria@study.thws.de & 9125072 \\
\hline
Katrina Alex & katrina.alex@study.thws.de & 9125071 \\
\hline
Manuel Stöth & manuel.stoeth@study.thws.de & 5123045 \\
\hline
Marvin Kraus & marvin.kraus@study.thws.de & 5123143 \\
\hline
\end{longtable}

\newpage

\section{Project Overview}

\subsection{Project Description}
WeatherNow Calendar Integration is a web-based application designed to simplify free-time planning by combining real-time weather data with the user’s personal calendar schedule. By leveraging the OpenWeatherMap API and the Google Calendar API, the app allows users to check current and forecasted weather conditions alongside their availability. Based on weather statistics and free days, WeatherNow suggests personalized activities—whether it’s a sunny outdoor hike or a cozy indoor museum visit. The frontend is built using React.js to ensure a modern, responsive, and smooth user experience across both mobile and desktop devices.

\subsection{Application Goals}
\begin{itemize}[nosep]
  \item Enable users to view accurate current weather and detailed forecasts for any location worldwide.
  \item Integrate the user’s Google Calendar to identify free and busy periods automatically.
  \item Suggest context-aware activities based on weather conditions and user availability.
  \item Provide a user-friendly, accessible, and visually appealing interface with light/dark mode options.
  \item Ensure fast, reliable, and error-tolerant performance for all weather and calendar interactions.
  \item Build a modular and scalable frontend that can be extended with additional APIs or features in the future.
\end{itemize}

\subsection{Target Users}
\begin{itemize}[nosep]
  \item \textbf{Travelers and Tourists} planning outdoor or sightseeing activities.
  \item \textbf{Daily Commuters} preparing for weather disruptions during workdays.
  \item \textbf{Outdoor Enthusiasts} (hikers, cyclists, sports players) relying on forecasts.
  \item \textbf{Busy Professionals and Students} maximizing free time with suitable recommendations.
  \item \textbf{Families and Groups} organizing weekend outings or vacations dependent on weather.
\end{itemize}

\newpage

\section{Project Complexity}
This project exhibits moderate complexity:
\begin{itemize}[nosep]
  \item \textbf{Multiple Views:} Home, Forecast, Calendar, Settings.
  \item \textbf{API Integration:} Asynchronous calls to two external services.
  \item \textbf{Interactive Widgets:} Calendar component, geolocation lookup.
  \item \textbf{State Management:} React Context (or Redux) for global state.
  \item \textbf{Responsive Design:} Mobile-first layout with CSS Grid/Flexbox.
\end{itemize}

\section{Key Features}

\subsection*{Must-Have}
\begin{itemize}[nosep]
  \item \textbf{City Search \& Current Weather:} Users search a city, see temperature, wind, humidity, conditions and icon.
  \item \textbf{5-Day Forecast:} Forecast in 3-hour intervals with daily high/low summaries.
  \item \textbf{Geolocation Lookup:} Automatically fetch weather for user’s current coordinates.
  \item \textbf{Unit Toggle:} Switch between °C and °F.
  \item \textbf{Theme Toggle:} Light and dark modes, persisted across sessions.
  \item \textbf{Interactive Calendar:} Select any date to trigger analysis.
  \item \textbf{Activity Suggestions:} Recommend indoor/outdoor activities based on weather rules.
  \item \textbf{Error Handling:} Friendly messages for invalid input, denied permissions, or API failures.
  \item \textbf{Responsive UI:} Adapts to mobile, tablet, and desktop breakpoints.
  \item \textbf{Client-Side Navigation:} React Router routes for Home, Forecast, and Calendar.
\end{itemize}

\subsection*{Nice-to-Have}
\begin{itemize}[nosep]
  \item Store and recall recently searched cities.
  \item Sync and overlay multiple Google calendars.
  \item Localize UI text and weather descriptions (e.g., EN/DE).
  \item Lookup historical weather for past dates.
  \item Push notifications for severe weather or upcoming free days.
\end{itemize}


\newpage


\section{User Roles and Interactions}

\subsection{Primary User Role}
The primary user of the application is the general user—an individual seeking to integrate real-time weather information with their personal calendar to make informed decisions about daily activities. This includes travelers, commuters, students, and outdoor enthusiasts who benefit from location-based and schedule-aware recommendations.

\subsubsection*{\textbf{Standard User}}
The Standard User primarily uses the app for quick access to current location-based and time-relevant information. Core functionalities include:
\begin{itemize}[nosep]
  \item Searching for a city and viewing current weather conditions such as temperature, wind, humidity, and weather icons.
  \item Accessing a 5-day forecast with 3-hour interval details and daily high/low summaries.
  \item Utilizing geolocation to automatically fetch weather data for the user’s current coordinates.
  \item Switching units between °C and °F.
  \item Toggling between light and dark themes, with preferences persisted across sessions.
  \item Quickly checking calendar availability to make spontaneous activity decisions based on current weather and free time.
\end{itemize}

\subsubsection*{\textbf{Planner}}
The Planner role focuses on proactively selecting and scheduling expeditions and activities based on weather forecasts and calendar availability. Core functionalities include:
\begin{itemize}[nosep]
  \item Browsing and filtering potential expeditions or activities according to detailed future weather forecasts and personal calendar events.
  \item Planning activities in advance by integrating forecast data with free/busy calendar periods to optimize scheduling.
  \item Managing and adjusting planned expeditions within the calendar interface.
  \item Leveraging the app’s personalized activity suggestions to maximize free time considering weather conditions.
\end{itemize}

\newpage

\subsection{Key User Interactions}
\begin{itemize}[nosep]
  \item \textbf{Search Weather by City:} Users can input or select a city to view its current weather conditions and forecast.
  \item \textbf{Grant Calendar Access:} Users may authorize the application to access their Google Calendar in order to detect available time slots.
  \item \textbf{Select a Calendar Date:} Clicking on a date in the calendar view fetches weather data for that day and displays personalized activity suggestions.
  \item \textbf{Toggle Display Settings:} Users can switch between temperature units (°C/°F) and choose between light and dark themes for better visual comfort.
  \item \textbf{Navigate Between Views:} Navigation between the Home, Forecast, and Calendar pages is supported through a fixed header or side menu.
  \item \textbf{Receive Feedback and Notifications:} The application provides real-time feedback through inline messages or toast notifications in case of input errors, denied permissions, or API issues.
\end{itemize}

\subsection{Example User Journey}
\begin{enumerate}[nosep]
  \item A user opens the application on their mobile browser and allows location access.
  \item The application fetches and displays the current weather for their location.
  \item The user switches the theme to dark mode for evening browsing and toggles the temperature to Fahrenheit.
  \item When the user tries an invalid city name, the app shows a helpful toast notification: “City not found. Please try again.”
  \item The user chooses between Planner and Standard Mode.
  \item Standard: The application retrieves their availability and weather forecast for that date.
  \item Standard: Based on sunny conditions and free time, the app suggests an outdoor cycling trip.
  \item Standard: The application shows a relevant split of the users calendar with his preplanned activities for the day.
  \item Planner: The application shows a two week calendar preview and presents quick access to weather information on each day.
  \item Planner: The user can search for suitable activities based on the location, time and consecutive Wheather.
\end{enumerate}

\newpage

\section{User Stories / Use Cases}
\begin{longtable}{|p{1cm}|p{3cm}|p{10cm}|}
\hline
\textbf{ID} & \textbf{Title} & \textbf{Description} \\
\hline
US1 & Search City & As a Primary User, I enter a city name to view current weather. This helps me check the weather anywhere in the world quickly and easily. \\
\hline
US2 & View Forecast & As a Planner, I request a five-day forecast to plan ahead. This allows me to make informed decisions about travel or outdoor activities in the coming days. \\
\hline
US3 & Use My Location & As a Primary User, I grant location access to see local weather. This saves time and provides accurate data based on my current GPS location. \\
\hline
US4 & Select Date & As a Planner, I click a calendar date to get weather stats and activity suggestions. This enables me to plan free-time activities according to the forecast. \\
\hline
US5 & Toggle Units & As a Primary User, I switch between Celsius and Fahrenheit. This ensures the app respects my regional preferences or personal comfort with temperature units. \\
\hline
US6 & Toggle Theme & As a Primary User, I enable dark mode at night. This reduces eye strain and improves usability in low-light environments. \\
\hline
US7 & Handle Error & As a Primary User, I get notified on API failures or invalid input. This helps me understand what went wrong and how I can fix it, such as retrying or correcting input. \\
\hline
US8 & Recent Searches & As a Standard User, I want to revisit recently searched cities for quick access. This improves efficiency by allowing me to check multiple cities I care about frequently. \\
\hline
US9 & Severe Weather Alert & As a Primary User, I want to be notified of severe weather to adjust my plans. This allows me to stay safe and react early to dangerous or disruptive conditions. \\
\hline
US10 & Multi-language Support & As a multilingual user, I want to view weather descriptions in my preferred language that is available. This improves accessibility and ensures clarity for non-English users. \\
\hline
US11 & Simple Layout & As a Standard User, I want to be presented with a simple layout, that shows me exactly the information relevant at this moment.
\hline
\end{longtable}

\newpage

\section{Non-Functional Requirements}
\begin{itemize}[nosep]
  \item \textbf{Usability:} Clear labels, consistent icons, minimal learning curve.
  \item \textbf{Responsiveness:} Fluid layout using CSS Flexbox/Grid for all viewports.
  \item \textbf{Accessibility:} ARIA labels, sufficient contrast, keyboard navigation.
  \item \textbf{Performance:} Data fetch and render complete under 1.5\,s on average networks.
  \item \textbf{Security:} HTTPS only; API keys in environment variables.
  \item \textbf{Maintainability:} Modular React components; documented code.
  \item \textbf{Scalability:} Architecture supports adding new APIs/features with minimal refactoring.
\end{itemize}



\section{Technology Assumptions}
\begin{itemize}[nosep]
  \item \textbf{Frontend:} React.js (ES6+).
  \item \textbf{Styling:} CSS Modules or Tailwind CSS.
  \item \textbf{Routing:} React Router DOM.
  \item \textbf{APIs:} OpenWeatherMap REST, Google Calendar OAuth2.
  \item \textbf{State Management:} React Context or Redux.
  \item \textbf{Build/Deploy:} Docker multi-stage build; Nginx serves static files.
\end{itemize}

\section{Project Constraints}
\begin{itemize}[nosep]
  \item \textbf{OpenWeatherMap API Limit:} The app uses the student tier of OpenWeatherMap, which allows up to 3,000 API calls per minute. Efficient usage and caching are required to stay within this limit.
  \item \textbf{Google Calendar Quotas \& OAuth:} Calendar integration is subject to Google’s API quotas and OAuth consent requirements. Users must authorize access, and rate limits must be respected.
  \item \textbf{Permission Dependency:} The app relies on user permission for location and calendar access. If denied, some features (e.g., weather-based suggestions) won’t function and must be handled gracefully.
  \item \textbf{Frontend-Only Architecture:} No custom backend will be developed. All logic, state management, and API calls are handled in the frontend, which limits secure data handling and processing.
  \item \textbf{Submission Deadline:} The Requirements Document is due on April 30, 2025, at 23:59, and sets the foundation for all following development phases.
\end{itemize}

\section{Acknowledgement of AI Tools}
This document was drafted and refined using GPT-4o based on an outline provided by the authors. The ideas of this document were partly generated by using GPT-4o and then integrated in the overall design vision. Furthermore was AI used to help with formatting


\end{document}
